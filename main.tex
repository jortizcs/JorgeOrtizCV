\documentclass[12pt]{article}

%% Define some new colors
\usepackage{xcolor}
\definecolor{mBlue}{RGB}{51, 77, 167}
\newcommand{\blue}[1]{{\color{mBlue}#1}}

%% Page and text formatting
\usepackage[left=1.0in, right=1.0in, top=1.0in, bottom=1.0in]{geometry} % margins
\usepackage{setspace}
\singlespacing % No more than 6 lines of text per inch
\usepackage{amsmath, amsfonts}
\usepackage[T1]{fontenc}
\usepackage{times}
\usepackage{lastpage}

%% Make the sections start with (a), (b), etc.
\renewcommand\thesection{(\alph{section})}
\renewcommand\thesubsection{(\alph{subsection})}

%% Set up footer
\usepackage{fancyhdr}
\fancypagestyle{CVfooter}
{
 \lhead{}
 \chead{}
 \rhead{}
 \lfoot{\small{Jorge J.\ Ortiz}}
 \cfoot{\small{\today}}
 \rfoot{\small{\thepage/\pageref{LastPage}}}
 \renewcommand{\headrulewidth}{0.0pt}
 \renewcommand{\footrulewidth}{0.5pt}
}

%% Set up hyperlinks
\usepackage[linktoc=page,colorlinks=true]{hyperref}
\hypersetup{
  urlcolor = mBlue,
  citecolor = mBlue,
  linkcolor = mBlue,
}

%% Set up citations and bibliography
\usepackage{bibunits}
\usepackage[sort&compress,super]{natbib}
\defaultbibliographystyle{apsrev4-2}
\setcitestyle{comma}
\setlength{\bibsep}{0pt}
\renewcommand{\bibnumfmt}[1]{\ \ \ #1.}
\renewcommand\refname{\vspace{-7mm}}

%% This allows us to start a bibliography with arbitrary number
\usepackage{etoolbox}
\makeatletter
\newcommand*{\newbibstartnumber}[1]{%
  \apptocmd{\thebibliography}{%
    \global\c@NAT@ctr #1\relax
    \addtocounter{NAT@ctr}{-1}%
  }{}{}%
}
\makeatother

%% Reduce whitespace around lists (globally)
\usepackage{enumitem}
\setlist[itemize]{nosep,left=0pt}
\setlist[enumerate]{nosep}
\setlist[description]{nosep}

%% Formatting for sections
\usepackage[compact]{titlesec}
\titleformat*{\section}{\normalsize\bfseries}
\titlespacing*{\section}{0mm}{2mm}{1mm}[0mm] % left top bottom right
\titleformat*{\subsection}{\normalsize\bfseries}
\titlespacing*{\subsection}{0mm}{2mm}{1mm}[0mm]
\titleformat*{\subsubsection}{\normalsize\bfseries\itshape}
\titlespacing*{\subsubsection}{0mm}{1mm}{0mm}[0mm]


%%%%%%%%%%%%%%%%%%%%%%%%%%%%%%%%%%%%%%%%%%%%%%%%%%%%%%%%%%%%


% Load bibentry after hyperref to avoid conflicts
\usepackage{bibentry}
\nobibliography*


% hanging publications: 1st line starts at beginning of line, 
% further lines are placed a bit to the right
\usepackage{hanging}
\newcommand\publication[1]{%
	\smallskip\par\hangpara{1.5em}{1}\bibentry{#1}\smallskip
}
%%%%%%%%%%%%%%%%%%%%%%%%%%%%%%%%%%%%%%%%%%%%%%%%%%%%%%%%%%%%








\begin{document}
\pagestyle{CVfooter}

\noindent \textbf{\uppercase{curriculum vitae}}
\vspace{2mm}

\noindent \textbf{Jorge J.\ Ortiz}
\vspace{1mm}

\noindent\begin{tabular*}{\textwidth}{@{\extracolsep{\fill}}l r}
Associate Professor (with Tenure)\\ Electrical and Computer Engineering & Email: \href{mailto:jortiz@alum.mit.edu}{jortiz@alum.mit.edu} \\
Rutgers University & Web: \href{http://jorgeortizphd.info}{http://jorgeortizphd.info} \\
94 Brett Road, & Phone: 617-784-6550 \\
Piscataway, NJ 08854-8058 &  \\
\hline
\end{tabular*}

\vspace{2mm}
\noindent \textbf{Research Focus:} Quantitative methods in machine learning, statistical modeling, optimization algorithms, and data-driven decision systems with applications to complex real-world problems. Extensive experience in mathematical modeling, algorithmic development, and large-scale data analysis across academic research and high-stakes industry applications.

\section{Education \& Training}

\begin{tabular}{l l l l}
University of California & Berkeley, CA & Computer Science & Ph.D., 2013 \\
University of California & Berkeley, CA & Computer Science & M.S., 2010 \\
M.I.T. & Cambridge, MA & Computer Science & B.S., 2003 \\
\end{tabular}

\section{Quantitative Expertise \& Technical Skills}

\textbf{Mathematical Modeling \& Optimization:} Advanced statistical modeling, convex and non-convex optimization, stochastic processes, time series analysis, Bayesian inference, Monte Carlo methods, and algorithmic trading strategies.

\textbf{Machine Learning \& AI:} Deep learning architectures (CNNs, RNNs, Transformers), reinforcement learning, ensemble methods, feature engineering, model selection and validation, hyperparameter optimization, and production ML systems.

\textbf{Data Analysis \& Computational Methods:} Large-scale data processing, distributed computing, high-performance computing, numerical methods, signal processing, multivariate statistics, and real-time analytics.

\textbf{Programming \& Technologies:} Python, R, MATLAB, C/C++, Java, SQL, NoSQL databases, cloud computing (AWS, GCP), distributed systems, and quantitative finance libraries.

\section{Academic Appointments}

\begin{tabular}{l l}
Sept. 2025 -- present & Associate Professor, Rutgers University \\
Sept. 2018 -- 2025 & Assistant Professor, Rutgers University \\
\end{tabular}

\section{Research Leadership}

\begin{tabular}{l l}
Sept. 2018 -- present & Director, Sensing and Reasoning (SnR) Lab, Rutgers University \\
\end{tabular}

\section{Industry Experience}

\begin{tabular}{l l}
Dec. 2019 -- present & \textbf{AI and Computer Vision Lead Baseball Ops., New York Yankees} \\
& \textit{Lead quantitative analyst developing predictive models, optimization} \\
& \textit{algorithms, and real-time decision systems for competitive advantage} \\
& \textit{in high-stakes professional sports environment} \\
Dec. 2013 -- Aug. 2018 & \textbf{Research Staff Member, IBM Research} \\
& \textit{Developed machine learning algorithms, optimization techniques,} \\
& \textit{and large-scale data analytics systems for enterprise applications} \\
Jan. 2013 -- Sept. 2013 & \textbf{Senior Software Engineer, Spire Global} \\
& \textit{Built mathematical models and algorithms for satellite data processing} \\
Aug. 2003 -- Feb. 2007 & \textbf{Software Engineer, Oracle Corp.} \\
& \textit{Developed database optimization and performance analytics systems} \\
\end{tabular}

\section{Honors and Invited Talks}
\begin{itemize}
    \item Invited Panelist at the JPMC Hispanic Heritage Month Celebration Panel on The Impact of Latinos and Technology, October 2, 2024, New York, NY.
    \item Invited Panelist at the Sports and AI Symposium, Columbia University, September 2024.
    \item Invited speaker at I-SENSE Distinguished Seminar Series, Florida Atlantic University, April 2024: \emph{Multimodal Learning and Sense-Making in Deeply Sensed Environments}
    \item Invited speaker at Samsara IoT Speaker Series, March 2024: \emph{Multimodal Learning: Application-Driven Design and Fundamentals}
    \item Invited speaker at HCII Seminar Series, Carnegie Mellon University, November 2023: \emph{Advancing Human-Machine Interactions: Multimodal Learning in Densely Sensed Spaces}
    \item Invited speaker at New Jersey Institute of Technology (NJIT) Computer Science Colloquium, November 2021
    \item Invited speaker at University of Massachusetts at Amherst (UMass Amherst) ECE Colloquium, October 2020
    \item Invited speaker at University of Southern California (USC), CPS-IoT Webinar, October 2020
    \item Invited speaker at Morris Meister Lecture Series, Bronx Science Foundation, Alumni Day, 2019
    \item Invited speaker at NSF Workshop on the Dynamic Interaction of Embodied Human and Machine Intelligence, May 2019
    \item Invited speaker ECE Seminar at Emory University, February 2019
    \item {\bf Keynote Speaker}: Workshop on Smart and Connected Indoor Environments \emph{in conjunction with IEEE International Conference on Sensing, Communication and Networking (SECON 2017)}
    \item Qualcomm Innovation Fellowship Finalist 2011
    \item NSF Graduate Fellowship Honorable Mention 2008
    \item Ford Foundation Diversity Fellowship Honorable Mention 2008
\end{itemize}

\section{Key Research Areas \& Impact}
\begin{itemize}
    \item \textbf{Machine Learning \& Sensing:} Developed algorithms for multimodal sensor fusion, human activity recognition, and real-time analytics systems with applications to smart environments and human behavior understanding.
    \item \textbf{Large-Scale Data Systems:} Led research in distributed computing architectures, IoT data processing systems, and intelligent building technologies for urban sensing applications.
    \item \textbf{Mathematical Modeling:} Applied advanced statistical methods, optimization techniques, and machine learning to human behavior prediction and sensing applications in various domains.
    \item \textbf{Sports Analytics:} Currently leading AI and computer vision initiatives for professional baseball operations, developing data-driven approaches for performance analysis and strategic decision support.
\end{itemize}

\section{Grants and Funding (\$32.4M Total)}
\begin{itemize}
    \item NSF IUCRC Phase I: Center on Responsible Artificial Intelligence and Governance (CRAIG), National Science Foundation Award \#2515224. Rutgers Site PI Jorge Ortiz. \$302,272 (2025-2030).
    \item NSF Planning Grant: AI Ready: STAIRWAI to COSMOS: Sensor-enabled Testbed for Advancing Innovative Research in Wireless+AI, National Science Foundation Award \#2509233. Co-Principal Investigator: Jorge Ortiz. \$200,000 (2025-2027).
    \item National Science Foundation Grant: ReDDDoT Phase 2: Leveraging Urban AI as a Communal Tool for Connection and Exchange in Harlem. Co-Principal Investigator: Jorge Ortiz (\$1,447,662).
    \item University Research Council Award: Toward next-generation ``intelligent'' pharmaceutical drug product manufacturing for efficient patient healthcare, Rutgers University. Co-Investigator: Jorge Ortiz (\$25,000)
    \item NIH Grant for Developing the Context-Aware Multimodal Ecological Research and Assessment (CAMERA) Platform, National Institute of Mental Health. Co-Investigator: Jorge Ortiz (\$1,079,407)
    \item NSF Engineering Research Center: The Center for Smart Streetscapes, National Science Foundation. Rutgers Site PI Jorge Ortiz (\$2.3M Rutgers). \$26,000,000.
    \item IUCRC Planning Grant Rutgers University: Center for Standards and Ethics in Artificial Intelligence (CSEAI), National Science Foundation Award Abstract \#2137245. PI Jorge Ortiz. \$20,000.
    \item Social Intelligence in the Automobile, Nissan Corporation. PI Jorge Ortiz. \$10,000.
    \item SII Planning: ARIES: Center for Agile, RelIablE, Scalable Spectrum, National Science Foundation. PI: Narayan Mandayam. Co-PI Jorge Ortiz (25\% effort). \$50,000.
\end{itemize}


\section{Selected Publications \& Patents}
\textbf{Key Publications \& Awards:}
\begin{itemize}
    \item \textbf{Best Paper Award:} ICISSP 2018 (21\% acceptance rate)
    \item \textbf{Best Paper Finalist:} BuildSys 2015 (23\% acceptance rate, 1st highest review score)
    \item \textbf{Best Paper Finalist:} BuildSys 2015 (23\% acceptance rate, 2nd highest review score)
    \item Publications in top-tier conferences: IPSN, SenSys, BuildSys, ICISSP, and others
    \item \textbf{Keynote Speaker:} Workshop on Smart and Connected Indoor Environments (SECON 2017)
\end{itemize}

\textbf{Patents \& Intellectual Property:}
\begin{itemize}
    \item 12+ issued patents in machine learning algorithms and data systems
    \item Patents in IoT data discovery, cloud computing, and recommendation systems
    \item Intellectual property in distributed computing and sensor data processing
\end{itemize}

\section{Past Ph.D. Students}
\begin{itemize}
    \item Tahiya Chowdhury. Data-Driven Techniques for Human Activity Sensing in Smart Environments. 08/2022.
    \item Murtadha Aldeer. Sensing and Machine Learning Techniques for Human Behavior Understanding through Physical Interaction. 05/2023.
    \item Tong Wu. Novel Methods for Predicting Timing and Attention in Human-agent Interaction through Application-driven Scenarios. 05/2023.
\end{itemize}



\section{Publications}
% bibliography
	\nobibliography{10_workshops.bib, 11_conf-journ.bib, 12_publications.bib, 20_under-submission.bib, 30_patents.bib}
% 	\bibliographystyle{apacite}	
	\bibliographystyle{acm}
	
	\section*{Under Submission}
	\begin{enumerate}
            % \item \publication{pargoo2024sass}
            % \item \publication{Sun2024}
            \item \publication{chowdhury2024designing}
	    % \item \publication{sun2024gexse}
            \item \publication{sun2024vchar}
            \item \publication{ortiz2024aligncap}
            \item \publication{ortiz2024sass}
	\end{enumerate}
	
	\section*{Conferences and Journals}
	\begin{enumerate}[topsep=0pt,itemsep=-1ex,partopsep=1ex,parsep=1ex]
            \item \publication{sun2025dfgauss}
            [\textbf{24.52\% Acceptance Rate}]
            \item \publication{pmlr-v205-ondras23a}
            [\textbf{30\% Acceptance Rate}]
            \item \publication{2023_10001054}
            \item \publication{2022_10.1145/3491245}
            \item \publication{2022_9804515}
            [\textbf{25\% Acceptance Rate}]
            \item \publication{2022_drive_interaction}
	    \item \publication{2021_ortiz_inagt_multisensor}
	    [\textbf{21\% Acceptance Rate}]
	    \item \publication{2021_aldeer_multibottle}
	    \item \publication{2021_toothbrushing_iot}
	    [\textbf{20\% Acceptance Rate}]
            \item \publication{2021_9686084}
            \item \publication{2021_9677368}
	    \item \publication{2021_9076254}
        [\textbf{30\% Acceptance Rate}]
        \item \publication{2020_10.1145/3417991}
        \item \publication{2019_10.1145/3360774.3360796}
        \item \publication{2019_10.1145/3302505.3310073}
            [\textbf{28\% Acceptance Rate, \textcolor{red}{Best Paper Finalist}}]
        \item \publication{2018_8489319}
        \item \publication{2018_10.1117/12.2306009}
        \item \publication{2018_icissp18}
            [\textbf{21\% Acceptance Rate, \textcolor{red}{Best Paper Award}}]
        \item \publication{2017_8260617}
            [\textbf{17\% Acceptance Rate}]
        \item \publication{2017_7953300}
        
        \item \publication{2016_10.1145/3007646.3007653}
            [\textbf{21\% Acceptance Rate}]
        \item \publication{2015_10.1145/2821650.2821657}
            [\textbf{23\% Acceptance Rate, \textcolor{red}{Best Paper Finalist (2nd Highest Review Score)}}]
        \item \publication{2015_10.1145/2821650.2821667}
            [\textbf{23\% Acceptance Rate, \textcolor{red}{Best Paper Finalist, 1st Highest Review Score}}]
        \item \publication{2015_190492}
            [\textbf{28\% Acceptance Rate}]
        \item \publication{2013_6917580}
            [\textbf{20\% Acceptance Rate}]
        \item \publication{2013_Msdas}
        \item \publication{2010_10.1145/1869983.1870003}
            [\textbf{17\% Acceptance Rate}]
        \item \publication{2010_10.1145/1791212.1791233}
            [\textbf{17\% Acceptance Rate}]
        \item \publication{2007_10.1145/1236360.1236383}
            [\textbf{22\% Acceptance Rate}]
    \end{enumerate}

	\section*{Workshops, Posters, and Technical Reports}	

    \begin{enumerate}[topsep=0pt,itemsep=-1ex,partopsep=1ex,parsep=1ex]
        \item \publication{2025_li_tellme}
        \item \publication{2025_xia_policygrid}
        \item \publication{2024_Grimalovsky}
        \item \publication{2024_Walter}
    
        \item \publication{2023_10.1145/3583120.3589836}
        \item \publication{2023_10.1145/3583120.3589839}
        \item \publication{2023_Aldeer_Testbed}

        \item \publication{2022_rivas2022planning}
        \item \publication{2022_chowdhury_cadence}
        \item \publication{2021_balances_buildsys}
        \item \publication{2021_fatesys_buildsys}
        \item \publication{2021_buildsys_poster_privacy_challenge}
        \item \publication{2021_wu2021rlad}
        \item \publication{2021_cphs01_maestro}
        \item \publication{2021_cphs02_pill}
        \item \publication{2019_10.1145/3360773.3360886}
        \item \publication{2019_10.1145/3360322.3361011}
        \item \publication{2019_10.1145/3356250.3361933}
        \item \publication{2018_Lee2018TimeSS}
        \item \publication{2016_10.1145/2993422.2996401}
        \item \publication{2016_icml_mkmeans}
        \item \publication{2016_7746808}
        \item \publication{2016_10.1145/2993422.2996398}
        \item \publication{2016_10.1145/2993422.2996401}
        \item \publication{2015_DBLP:journals/corr/HongOBW15} 
        \item \publication{2014_10.1145/2674061.2675031}
        \item \publication{2014_10.1145/2668332.2668369}
        \item \publication{2014_Bhattacharya:EECS-2014-159}
        \item \publication{2013_10.1145/2528282.2528302}
        \item \publication{2012_10.1145/2422531.2422540}
        \item \publication{2012_sbs:workshop2012}
        \item \publication{2011_KATZ20117}
        \item \publication{2011_6476843}
        \item \publication{2010_10.1145/1978642.1978648}
        \item \publication{2010_10.1145/1878431.1878444}
        \item \publication{2010_Ortiz:EECS-2010-128}
        \item \publication{2008_10.1145/1460412.1460478}
            [\textbf{\textcolor{red}{Best Poster Award}}]
        \item \publication{2008_Dawson-Haggerty:EECS-2008-109}
        \item \publication{2007_10.1145/1317103.1317112}
        \item \publication{2007_10.1145/1317103.1317109}
    \end{enumerate}

    \section*{Books and Book Chapters}
    \begin{enumerate}
        \item \publication{2022_aldeer2022monitoring} 
    \end{enumerate}
    
    
    \section*{Ph.D. Thesis}
    \begin{enumerate}
        \item \publication{2013_Ortiz:EECS-2013-196}
    \end{enumerate}
    
    \section*{Masters Thesis}
    \begin{enumerate}
        \item \publication{2010_Ortiz:EECS-2010-65}
    \end{enumerate}
    
    \section*{Patents Issued}
    \begin{enumerate}
        \item \publication{patent_13}
        \item \publication{patent_12}
        \item \publication{patent_11}
        \item \publication{patent_10}
        \item \publication{patent_9}
        \item \publication{patent_8}
        \item \publication{patent_7}
        \item \publication{patent_6}
        \item \publication{patent_5}
        \item \publication{patent_4}
        \item \publication{patent_3}
        \item \publication{patent_2}
        \item \publication{patent_1}
    \end{enumerate}

\newpage
\section{Service}
\begin{enumerate}
    \item Judge, Newsweek AI Impact Awards 2025 (\href{https://events.newsweek.com/aiimpact-us/awards-us-2025}{https://events.newsweek.com/aiimpact-us/awards-us-2025})
    \item Instructor, Electrical and Computer Engineering Component, Rutgers Honors Engineering Experience (RHEx) Summer Program 2025
    \item TPC Member ACM MobiHoc 2025
    \item TPC Member Sensys 2025
    \item TPC Member Buildsys 2025
    \item Reviewer LatinX in AI Supercomputing Network Cohort II (LXAISN) 2025
    \item TPC Member ICLR 2025
    \item Steering Committee Chair BuildSys 2024-2025
    \item IPSN Poster co-Chair 2023
    \item General Chair Buildsys 2022
    \item Chair for Workshop on Cyber-Physical-Human Systems (CPHS) at CPS Week 2022
    \item NeurIPS 2021 Workshop on The Symbiosis of Deep Learning and Differential Equations (DLDE), Organizer
    \item AAAI-22 Undergraduate Consortium -- Mentorship Coordinator Chair
    \item TPC The First International Workshop on Cyber-Physical-Human System Design and Implementation at CPS Week 2021
    \item EWSN 2021 Poster Co-Chair
    \item Buildsys 2020 TPC Co-Chair, Steering Committee 2021, TPC 2021, Poster Chair 2021
    \item AAAI-21 Undergraduate Consortium -- Faculty Mentor
    \item AAAI-20 Undergraduate Consortium -- Faculty Mentor
    \item LatinX in AI Mentor 2020
    % \item New York Scientific Data Summit (NYSDS) 2020 Co-Chair
    \item Co-Chair N2Women at Sensys 2019
    \item Co-Chair Workshop on Combining Physical and Data-Driven Knowledge in Ubiquitous Computing at UBICOMP 2019
    \item CPS-IoT Week 2019 Publication Chair (HSCC, ICCPS, IPSN, RTAS, IoTDI)
    \item Richard Tapia Celebration in Diversity in Computing Poster Co-Chair 2018, Chair 2019
    \item TPC Member Latinos in Artificial Intelligence Workshop NeurIPS 2018, ICML 2019
    \item TPC Member Energy Data and Analytics Workshop at e-Energy 2018, 2019
    \item TPC Member Sensys 2016 (Demo Session), Organizing Committee 2020
    \item TPC Member IPSN 2014, 2017, 2019, 2020, 2021
    \item TPC Member Buildsys 2016, 2017, 2018, 2019, 2020
    \item Organizing Committee Sensys/Buildsys 2016, 2017, 2020
    \item The 26th International Conference on Computer Communication and Networks (ICCCN 2017)
    \item TPC Member 1st Workshop on Smart and Connected Indoor Environments (SCIE) in conjunction with IEEE International Conference on Sensing, Communication and Networking (SECON 2017)
    \item TPC Member 1st Workshop on Internet of Thing Physical Data Analytics (IoTPDA) in conjunction with IEEE International Conference on Sensing, Communication and Networking (SECON 2016)
    \item TPC Member DCOSS 2016 (The annual International Conference on
    Distributed Computing in Sensor Systems)
    \item TPC Member ALGOSENSORS 2017 
    \item TPC IEEE workshop on Big Data Management for the Internet of Things (BIOT2017)
    \item Member of CSGSA Faculty Candidate Evaluation Committee, UCB Computer Science Div. (2008, 2009, 2010)
    \item President of CSGSA Faculty Candidate Evaluation Committee, UCB Computer Science Div. (2011)
    \item Chair, Internet of Things Professional Interest Community (IoT-PIC) IBM Research at Watson Labs.
\end{enumerate}



\end{document}

