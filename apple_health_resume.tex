\documentclass[11pt,letterpaper]{article}

% Professional packages and setup
\usepackage[utf8]{inputenc}
\usepackage[T1]{fontenc}
\usepackage{lmodern}
\usepackage{microtype}

% Colors - Apple-inspired subtle palette
\usepackage{xcolor}
\definecolor{appleGray}{RGB}{29, 29, 31}
\definecolor{accentGray}{RGB}{86, 86, 90}
\definecolor{lightGray}{RGB}{142, 142, 147}
\definecolor{subtleBlue}{RGB}{0, 122, 255}

% Page layout
\usepackage[letterpaper, margin=0.7in, top=0.55in, bottom=0.55in]{geometry}
\usepackage{setspace}
\setstretch{1.0}

% Typography and formatting
\usepackage[scaled=0.92]{helvet}
\renewcommand\familydefault{\sfdefault}
\usepackage{titlesec}
\usepackage{enumitem}
\usepackage{tabularx}
\usepackage{array}

% Hyperlinks
\usepackage[hidelinks]{hyperref}
\hypersetup{
    colorlinks=true,
    linkcolor=subtleBlue,
    urlcolor=subtleBlue,
    citecolor=subtleBlue
}

% No page numbers
\pagestyle{empty}

% Section formatting
\titleformat{\section}
{\large\bfseries\color{appleGray}}
{}
{0em}
{}[\vspace{1pt}{\color{lightGray}\titlerule[0.8pt]}\vspace{4pt}]

\titlespacing*{\section}{0pt}{8pt}{3pt}

% Subsection formatting
\titleformat{\subsection}
{\normalsize\bfseries\color{appleGray}}
{}
{0em}
{}

\titlespacing*{\subsection}{0pt}{5pt}{1pt}

% Custom commands
\newcommand{\resumeentry}[4]{
    \noindent
    \begin{minipage}[t]{0.62\textwidth}
        \textbf{\color{appleGray}#1}\\
        \textcolor{accentGray}{#3}
    \end{minipage}%
    \hfill
    \begin{minipage}[t]{0.36\textwidth}
        \raggedleft
        \textcolor{accentGray}{\textit{#2}}\\
        #4
    \end{minipage}
    \vspace{2pt}
}

\newcommand{\simpleentry}[2]{
    \begin{tabularx}{\textwidth}{@{}X r@{}}
        \textbf{\color{appleGray}#1} & \textcolor{accentGray}{#2}
    \end{tabularx}
    \vspace{1pt}
}

% Custom lists
\setlist[itemize]{leftmargin=12pt, itemsep=1pt, parsep=0pt, topsep=1pt}

\begin{document}

% Header
\begin{center}
{\fontsize{26}{30}\selectfont\bfseries\color{appleGray}JORGE J. ORTIZ}\\[3pt]
{\large\color{accentGray}Associate Professor, Computer Science \& Engineering}\\[1pt]
{\normalsize\color{accentGray}Machine Learning Researcher \textbullet\ Health AI \& Biosensor Systems}\\[4pt]

\begin{tabular}{c@{\hspace{12pt}}c@{\hspace{12pt}}c@{\hspace{12pt}}c}
\color{accentGray}{New Brunswick, NJ} &
\href{mailto:jorge.ortiz@rutgers.edu}{\color{subtleBlue}{jorge.ortiz@rutgers.edu}} &
\href{http://jorgeortizphd.info}{\color{subtleBlue}{jorgeortizphd.info}} &
\color{accentGray}{617-784-6550}
\end{tabular}
\end{center}

\vspace{2pt}

\section*{PROFESSIONAL SUMMARY}

Associate Professor with 10+ years designing causal and multimodal AI systems for health sensing. Built large-scale ML pipelines for wearable and ambient biosignals, including production models for the New York Yankees and NIH-funded mental health research processing 10M+ sensor samples weekly. Published 40+ peer-reviewed papers in premier venues (NeurIPS, CoRL, ACM BuildSys) with expertise translating research into production systems. Planning sabbatical (July 2026 -- July 2027) to contribute to Apple's AIML Health research program. Interested in advancing foundation models for health and human performance at Apple.

\textbf{Available for one-year full-time residency: July 13, 2026 -- July 9, 2027}

\section*{EDUCATION}

\simpleentry{Ph.D., Computer Science}{University of California, Berkeley, 2013}
\simpleentry{M.S., Computer Science}{University of California, Berkeley, 2010}
\simpleentry{B.S., Computer Science}{Massachusetts Institute of Technology, 2003}

\section*{HEALTH AI RESEARCH EXPERIENCE}

\subsection*{NIH CAMERA Platform — Context-Aware Multimodal Ecological Research and Assessment}
\textit{Principal Investigator (multi-PI), \$5.4M | NIH R61/R33MH135405} \hfill \textit{2024 -- 2029}
\begin{itemize}
    \item Leading development of multimodal research platform integrating neural, physiological, behavioral, and environmental sensors for continuous prediction of anxiety state and memory performance
    \item Developing causal and multimodal machine learning models to understand brain-behavior relationships and predict mental health states; published causal inference methods in NeurIPS workshops (PolicyGrid, TellMe)
    \item Designing novel algorithms for sensor fusion, context-aware ecological momentary assessments, and interpretable machine learning
    \item Collaborating with clinical researchers at Columbia University to validate health biomarkers and translate findings to practice
    \item Managing cross-functional team of engineers, neuroscientists, clinicians, and data scientists
\end{itemize}

\subsection*{PatientSense — Smart Biosensor System for Medication Monitoring}
\textit{Principal Investigator} \hfill \textit{2018 -- 2020}
\begin{itemize}
    \item Developed sensor-equipped pill-bottle system using IMU biosensors (similar sensing modalities to Apple Watch) for patient identification and medication adherence monitoring
    \item Achieved 90\%+ accuracy in discriminating individuals using inertial sensor data and machine learning
    \item Published in ACM MobiQuitous 2019; technology applicable to remote patient monitoring and medication safety
\end{itemize}

\subsection*{Sleep Monitoring \& Health Sensing Research}
\textit{Lead Researcher} \hfill \textit{2019 -- 2022}
\begin{itemize}
    \item Research on non-invasive sleep monitoring using wearable and ambient sensors; published authoritative review in ACM Transactions on Computing for Healthcare (2022)
    \item Investigated sleep stage classification, vital signs monitoring, and sleep disorder detection
\end{itemize}

\subsection*{MedBuds — In-Ear Biosensor for Medication Detection}
\textit{Co-Investigator} \hfill \textit{2021 -- 2022}
\begin{itemize}
    \item Developed medication-taking detection system using IMU sensors in wireless earbuds (similar form factor to AirPods); demonstrated feasibility of repurposing consumer audio devices for health monitoring (IEEE CPHS 2022)
\end{itemize}

\section*{ACADEMIC \& INDUSTRY EXPERIENCE}

\resumeentry{Associate Professor (with Tenure)}{July 2025 -- Present}{Rutgers University, Electrical \& Computer Engineering}{}

\resumeentry{Assistant Professor}{Sept. 2018 -- June 2025}{Rutgers University, Electrical \& Computer Engineering}{}

\resumeentry{Director, Sensing and Reasoning Lab}{Sept. 2018 -- Present}{Rutgers University}{Director of Sensing and Reasoning Lab, supervising 20+ active researchers (Ph.D., M.S., and undergraduate).}

\resumeentry{AI \& Computer Vision Lead}{Dec. 2019 -- Present}{New York Yankees, Baseball Operations}{Multimodal biomechanics analytics; causal modeling}

\resumeentry{Research Staff Member}{Dec. 2013 -- Aug. 2018}{IBM Research}{IoT and ML research}

\section*{TECHNICAL EXPERTISE}

\textbf{Programming Languages:} Python, C++, R, Objective-C, Swift, MATLAB\\[2pt]
\textbf{Machine Learning:} Causal ML, multimodal learning, foundation models, on-device ML, privacy-preserving modeling, computer vision, deep learning\\[2pt]
\textbf{Health \& Biosensors:} IMU sensors, wearable devices, PPG sensing, bio-signal processing, physiological signal analysis, health biomarker development\\[2pt]
\textbf{Software Engineering:} Large-scale ML systems, real-time sensing applications, edge computing, distributed systems\\[2pt]
\textbf{Research Impact:} 40+ peer-reviewed publications, 12+ issued patents, multiple best paper awards

\section*{SELECTED HEALTH-FOCUSED PUBLICATIONS}

\begin{itemize}
    \item Z. Hussain, Q. Z. Sheng, W. E. Zhang, \textbf{J. Ortiz}, and S. Pouriyeh, ``Non-invasive Techniques for Monitoring Different Aspects of Sleep: A Comprehensive Review,'' \textit{ACM Trans. Comput. Healthcare} 3, 2, Article 24 (April 2022), 26 pages. \href{https://doi.org/10.1145/3491245}{https://doi.org/10.1145/3491245}
    \item M. Aldeer, D. Waterworth, et al., \textbf{J. Ortiz}, ``MedBuds: In-Ear Inertial Medication Taking Detection Using Smart Wireless Earbuds,'' \textit{IEEE CPHS}, 2022.
    \item M. Aldeer, \textbf{J. Ortiz}, et al., ``Unobtrusive Patient Identification Using Smart Pill-Bottle Systems,'' \textit{Information Processing and Management}, 2021.
\end{itemize}

\section*{LEADERSHIP \& IMPACT}

\textbf{Team Leadership:} Director of 30+ researcher lab; mentored 3 Ph.D. graduates in causal ML and wearable sensing\\[2pt]
\textbf{Cross-Functional Delivery:} 5 years IBM Research; AI Lead NY Yankees; extensive academic-industry partnerships\\[2pt]
\textbf{Research Impact:} Best Paper Award (ICISSP 2018); Finalist recognitions at ACM IoTDI, BuildSys

\section*{RESEARCH FUNDING}

\textbf{NIH CAMERA Platform}, PI (\$5.4M) — Mental health AI using multimodal sensing | \textbf{NSF ERC}, Site PI (\$26M) — Urban AI | \textbf{NSF IUCRC}, PI (\$1.5M) — Responsible AI

\vspace{4pt}
\begin{center}
\textit{\color{accentGray}Available for AIML Residency program July 13, 2026 -- July 9, 2027 during planned sabbatical}
\end{center}

\end{document}

