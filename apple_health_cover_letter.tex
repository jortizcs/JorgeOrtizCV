\documentclass[11pt,letterpaper]{article}

% Professional packages and setup
\usepackage[utf8]{inputenc}
\usepackage[T1]{fontenc}
\usepackage{lmodern}
\usepackage{microtype}

% Colors - Apple-inspired subtle palette
\usepackage{xcolor}
\definecolor{appleGray}{RGB}{29, 29, 31}
\definecolor{accentGray}{RGB}{86, 86, 90}
\definecolor{subtleBlue}{RGB}{0, 122, 255}

% Page layout
\usepackage[letterpaper, margin=1in, top=0.75in, bottom=0.75in]{geometry}
\usepackage{setspace}
\setstretch{1.1}

% Typography and formatting
\usepackage[scaled=0.92]{helvet}
\renewcommand\familydefault{\sfdefault}

% Hyperlinks
\usepackage[hidelinks]{hyperref}
\hypersetup{
    colorlinks=true,
    linkcolor=subtleBlue,
    urlcolor=subtleBlue,
    citecolor=subtleBlue
}

% No page numbers
\pagestyle{empty}

\begin{document}

% Header - matching resume style
\begin{center}
{\fontsize{22}{26}\selectfont\bfseries\color{appleGray}JORGE J. ORTIZ, Ph.D.}\\[3pt]
{\normalsize\color{accentGray}Associate Professor (with Tenure), Electrical \& Computer Engineering}\\[1pt]
{\normalsize\color{accentGray}Rutgers University}\\[6pt]

\begin{tabular}{c@{\hspace{20pt}}c@{\hspace{20pt}}c}
\href{mailto:jorge.ortiz@rutgers.edu}{\color{subtleBlue}{jorge.ortiz@rutgers.edu}} &
\href{http://jorgeortizphd.info}{\color{subtleBlue}{jorgeortizphd.info}} &
\color{accentGray}{617-784-6550}
\end{tabular}
\end{center}

\vspace{18pt}

\noindent
\today

\vspace{12pt}

\noindent
Apple AIML Residency Program\\
Machine Learning and AI\\
Cupertino, CA

\vspace{12pt}

\noindent
\textbf{\color{appleGray}Re: Application for AIML Resident -- Health (Role Number: 200630790-2459)}

\vspace{16pt}

\noindent
Dear Apple AIML Residency Hiring Team,

\vspace{6pt}

I am writing to express my strong interest in the AIML Resident -- Health position. As an Associate Professor with tenure at Rutgers University, I am planning a sabbatical for the 2026-2027 academic year specifically to pursue this opportunity and contribute to Apple's efforts in foundation models for biosignals and health AI. My decade of research in multimodal AI, health sensing systems, and biosensor algorithms, combined with my experience translating research into real-world applications, aligns exceptionally well with Apple's mission to revolutionize health through machine learning.

\vspace{10pt}

\noindent
\textbf{\large\color{appleGray}My Research Interests in Health AI}

\vspace{8pt}

My research focuses on causal machine learning and designing foundation models that integrate wearable, ambient, and physiological signals. At Rutgers, I lead the NIH CAMERA project developing causal models that predict cognitive state from biosignals—work that parallels Apple's efforts in foundation models for health. I bring a decade of experience building multimodal health systems that ship: from IMU-based medication tracking to mental health prediction platforms processing millions of sensor readings weekly.

My work emphasizes privacy-preserving health AI and the ethical implications of deploying sensing systems at scale. I span the full pipeline from sensor design to clinical validation. The most impactful health AI emerges from understanding both the technical constraints of biosensors and the practical needs of users.

\vspace{10pt}

\noindent
\textbf{\large\color{appleGray}Relevance to the AIML Residency -- Health Position}

\vspace{8pt}

My experience directly addresses this residency's focus: causal ML for biosensors, multimodal learning, and health biomarker evaluation. I'm especially interested in how the Health Technologies group leverages PPG and motion data from Apple Watch for cardiometabolic and mental health research.

Through NIH CAMERA, I develop causal models for brain-behavior relationships, resulting in recent NeurIPS workshop publications (PolicyGrid, TellMe). This directly parallels Apple's ongoing research in multimodal modeling for mental health and human performance. My PatientSense work achieved 90\%+ accuracy identifying individuals from IMU sensors in everyday objects, while MedBuds detected medication-taking behaviors using inertial sensors in wireless earbuds—directly relevant to extracting health biomarkers from Apple devices.

I have published at NeurIPS, CoRL, and top systems conferences (ACM BuildSys, SenSys) with multiple best paper recognitions. My comprehensive sleep monitoring review in \textit{ACM Transactions on Computing for Healthcare} has become a field reference.

My five years at IBM Research and ongoing role as AI Lead for the New York Yankees provide extensive experience translating research into production systems. I understand how to balance research rigor with practical constraints and deliver impactful results in cross-functional teams.

\vspace{10pt}

\noindent
\textbf{\large\color{appleGray}Why the AIML Residency Program}

\vspace{8pt}

I want to bring my decade of experience building causal, multimodal health systems into Apple's research-to-product environment, and to learn how large-scale, privacy-preserving health AI is developed for millions of users. Apple's position at the intersection of consumer hardware and health AI offers an unprecedented opportunity to impact real-world health outcomes at scale.

I bring expertise in biosensors, causal ML, and multimodal learning. I want to deepen my understanding of foundation models for physiological signals and Apple's approaches to health biomarker development. The mentorship structure and ML/AI courses provide the ideal learning environment.

This residency represents the bridge between my academic research and real-world health product development. Understanding how Apple translates cutting-edge research into products that maintain research rigor while delivering exceptional user experiences will strengthen my ability to prepare students for impactful careers.

I'm eager to bring my expertise in causal and multimodal ML to Apple's health AI efforts and to collaborate with engineers and researchers advancing foundation models for biosignals. I share Apple's belief that advancing health technology responsibly can empower millions of users while protecting their privacy. I'm confident I can contribute from day one and gain the perspective needed to translate Apple-scale insights back into academia. I am fully committed to the complete residency program from July 13, 2026, through July 9, 2027, during my planned sabbatical.

\vspace{14pt}

\noindent
Sincerely,

\vspace{20pt}

\noindent
{\color{appleGray}\textbf{Jorge J. Ortiz, Ph.D.}}\\
Associate Professor (with Tenure)\\
Rutgers University

\end{document}

